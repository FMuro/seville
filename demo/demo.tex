\documentclass{beamer}
\usetheme{seville}
\usepackage{tikz-cd}

\addbibresource{demo.bib}

\title{Seville, a gorgeous beamer theme}
\subtitle{That was the title and this is the subtitle}
\author{John Doe}
\date{Conference Presentation \the\year{}}
\institute{University of \LaTeX}
\titlegraphic{\includegraphics{giralda.png}}
\email{john@doe.com}
\site[https://tug.org]{tug.org}
\arxiv{https://arxiv.org}
\github{GitHub}
\twitter{Twitter}

\begin{document}

\begin{frame}

    \titlepage

\end{frame}

\section{Introduction}

\begin{frame}
    \frametitle{Seville looks}

    Seville is a beamer theme inspired by Matthias Vogelgesang's beautiful Metropolis theme.

    This theme uses the Fira Sans font by Mozilla \faFirefoxBrowser, the Font Awesome 5 icons \faFontAwesome, and the Academicons \aiAcademiaSquare.

    The logo is borrowed from Graficatessen.

    Colors are taken from the Solarized palette \faPalette.

    Text can be \alert{alerted}, \textbf{bold}, or \emph{emphasized}.

    Presentations using this theme must be compiled with Lua\LaTeX.

\end{frame}

\section{Blocks}

\begin{frame}
    \frametitle{Beamer blocks\footnote{There are also predefined math block environments: \emph{definition}, \emph{example}, \emph{theorem}, \emph{proof}, \emph{corollary}, \emph{lemma}, \emph{fact}, \emph{proposition}, and \emph{remark}.}}

    \begin{block}{Block}
        This is the look of a normal beamer block.
    \end{block}

    \begin{alertblock}{Alert!}
        This is an alerted block.
    \end{alertblock}

    \begin{exampleblock}{Example}
        This is how an example block looks like with this theme.
    \end{exampleblock}

\end{frame}

\section{Lists}

\begin{frame}
    \frametitle{Lists}

    We have lists, with numbers or symbols, and three indentation levels.

    \begin{columns}
        \begin{column}{0.5\textwidth}
            \begin{enumerate}
                \item Carrots.
                      \begin{enumerate}
                          \item Orange.
                                \begin{enumerate}
                                    \item Long.
                                    \item Short.
                                \end{enumerate}
                          \item Purple.
                      \end{enumerate}
                \item Onions.
                \item Lettuce.
            \end{enumerate}
        \end{column}
        \begin{column}{0.5\textwidth}  %%<--- here
            \begin{itemize}
                \item Carrots.
                      \begin{itemize}
                          \item Orange.
                                \begin{itemize}
                                    \item Long.
                                    \item Short.
                                \end{itemize}
                          \item Purple.
                      \end{itemize}
                \item Onions.
                \item Lettuce.
            \end{itemize}
        \end{column}
    \end{columns}

\end{frame}

\section{References and citations}

\subsection{How to cite}

\begin{frame}
    \frametitle{Citations}

    Citations like \cite{knuth-fa} contain links to the reference list. Click on it!

    You can also credit theorems with citations.

    \begin{theorem}[\cite{einstein}]
        This theorem was proved by Einstein. Click on the red citation too!
    \end{theorem}

\end{frame}

\subsection{Reference list}

\begin{frame}
    \frametitle{References}
    \framesubtitle{Click on titles!}
    \nocite{*}
    \printbibliography[heading=none]
\end{frame}

\end{document}