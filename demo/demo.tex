\documentclass{beamer}
\usetheme{seville}

\title{Seville, a gorgeous beamer theme}
\subtitle{That was the title and this is the subtitle}
\author{John Doe}
\date{Conference Presentation \the\year{}}
\institute{University of \LaTeX}
\titlegraphic{\includegraphics{giralda.png}}
\email{john@doe.com}
\site[https://tug.org]{tug.org}
% use the [academicons] option before uncovering the arXiv link
%\arxiv{https://arxiv.org}
\github{FMuro/seville/}
\twitter{TeXUsersGroup}

\begin{document}

\begin{frame}

    \titlepage

\end{frame}

\section{Introduction}

\begin{frame}
    \frametitle{Seville looks}

    Seville is a beamer theme inspired by Matthias Vogelgesang's beautiful Metropolis theme.

    This theme uses the Font Awesome 5 icons \faFontAwesome{}.

    The logo is borrowed from Graficatessen.

    Colors are taken from the Solarized palette \faPalette.

    Text can be \alert{alerted}, \textbf{bold}, \emph{emphasized}, or \texttt{monospaced}.

    The default font is Adobe's Source Sans Pro with Source Code Pro monospaced font.

    Optionally, this theme can use the Alegreya Sans font by Huerta Tipográfica, the Fira Sans font by Mozilla \faFirefox, or the Noto Sans font by Google \faGoogle. Also the Academicons.

\end{frame}

\section{Blocks}

\begin{frame}
    \frametitle{Beamer blocks\footnote{There are also predefined math block environments: \emph{definition}, \emph{example}, \emph{theorem}, \emph{proof}, \emph{corollary}, \emph{lemma}, \emph{fact}, \emph{proposition}, and \emph{remark}.}}

    \begin{block}{Block}
        This is the look of a normal beamer block.
    \end{block}

    \begin{alertblock}{Alert!}
        This is an alerted block.
    \end{alertblock}

    \begin{exampleblock}{Example}
        This is how an example block looks like with this theme.
    \end{exampleblock}

\end{frame}

\section*{Math}

\begin{frame}
    \frametitle{Math symbols}

    Math symbols look as follows:
    \begin{align*}
        F(x)                                                   & =\int_{-\infty}^x\frac{1}{\sigma\sqrt{2\pi}}e^{-\frac{1}{2}\left(\frac{x-\mu}{\sigma}\right)^2}, &
        A\cap\bigcup_{n=0}^\infty B_i                          & =\bigcup_{n=0}^\infty (A\cap B_i),                                                                 \\
        f(x)                                                   & =\sum_{n=0}^\infty f'(a)\frac{(x-a)^n}{n!},                                                      &
        A\cup\bigcap_{n=0}^\infty B_i                          & =\bigcap_{n=0}^\infty (A\cup B_i),                                                                 \\
        A                                                      & = \begin{pmatrix}
                                                                       a_{11} & \cdots & a_{1p} \\
                                                                       \vdots & \ddots & \vdots \\
                                                                       a_{n1} & \cdots & a_{np}
                                                                   \end{pmatrix},                                                                      &
        X\otimes(Y\oplus Z)                                    & =X\otimes Y\oplus X\otimes Z,                                                                      \\
        \bigotimes_{i=1}^nA_i                                  & =A_1\otimes\cdots\otimes A_n,                                                                    &
        \operatorname{Hom}\left(\bigoplus_{i\in I}X_i,Y\right) &
        =\prod_{i\in I}\operatorname{Hom}(X_i,Y).
    \end{align*}

\end{frame}

\section{Lists}

\begin{frame}
    \frametitle{Lists}

    We have lists, with numbers or symbols, and three indentation levels.

    \begin{columns}
        \begin{column}{0.5\textwidth}
            \begin{enumerate}
                \item Carrots.
                      \begin{enumerate}
                          \item Orange.
                                \begin{enumerate}
                                    \item Long.
                                    \item Short.
                                \end{enumerate}
                          \item Purple.
                      \end{enumerate}
                \item Onions.
                \item Lettuce.
            \end{enumerate}
        \end{column}
        \begin{column}{0.5\textwidth}  %%<--- here
            \begin{itemize}
                \item Carrots.
                      \begin{itemize}
                          \item Orange.
                                \begin{itemize}
                                    \item Long.
                                    \item Short.
                                \end{itemize}
                          \item Purple.
                      \end{itemize}
                \item Onions.
                \item Lettuce.
            \end{itemize}
        \end{column}
    \end{columns}

\end{frame}

\section{References and citations}

\subsection{How to cite}

\begin{frame}
    \frametitle{Citations}

    Citations like \cite{knuth-fa} contain links to the reference list. Click on it!

    It also works with several papers in the same citation command, like \cite{dirac,knuthwebsite}.

    You can also credit theorems with citations.

    \begin{theorem}[\cite{einstein}]
        This theorem was proved by Einstein. Click on the red citation!
    \end{theorem}

\end{frame}

\subsection{Reference list}

\begin{frame}
    \frametitle{References}

    \bibliographystyle{apalike}
    \bibliography{demo}

\end{frame}

\end{document}